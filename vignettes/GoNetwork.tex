\documentclass{article}
\title{Using GoNetwork}
\author{Sara Linker}

\usepackage{Sweave}
\begin{document}
\maketitle
\Sconcordance{concordance:GoNetwork.tex:GoNetwork.Rnw:%
1 15 1 1 4 4 1 1 2 1 0 1 1 7 0 1 2 2 1 1 2 4 0 1 2 4 1 1 2 4 0 1 2 3 1 %
1 2 4 0 1 2 3 1 1 2 4 0 1 2 3 1 1 2 1 0 2 1 17 0 1 2 2 1 1 2 1 0 1 1 6 %
0 1 2 1 1}


\section*{Introduction}
  The purpose of GoNetwork is to calculate networks based on GO annotation and to identify clusters that are differentially enriched between conditions based on expression data. Networks created using this program are easily plotted with molecular visualization software platforms such as Cytoscape. }

# \section*{Background}
#   Functional annotation is often used as a descriptive tool to examine gene sets identified through upstream analyses. This has been useful in describing what types of pathways and mechanisms are attributable to a given system, however there are limitations in both the ability to describe a system and to quantify these descriptions with many of the current functional annotation methods. GONetwork is a user-friendly method that increases the information returned from functional enrichment analyses by reducing the reliance on }


\section*{Run}
GONetwork is designed to take a list of genes and calculate a similarity matrix based on the hierarchical representation of GO terms and taking into account the sparsity of the dataset. We've provided a test dataset to work with generated from real data examining neural progenitor cells throughout differentiation into neurons. The dataset \textsl{differentiation} contains an object \textsl{genes} with the significant genes from this analysis.

\begin{Schunk}
\begin{Sinput}
> data(differentiation)
> head(genes)
\end{Sinput}
\begin{Soutput}
[1] "GTF3C2-AS1" "PRMT5-AS1"  "CIDEB"      "CTRL"       "NUCB1-AS1" 
[6] "CERS6-AS1" 
\end{Soutput}
\end{Schunk}

\section*{Generate GOterm Matrix}
Use \textbf{getGo()} to create a matrix of GO terms based on the gene list.  GoNetwork improves upon previous functional gene enrichment tools by applying weights to the binary GO term matrix based on the number of parents in the GO hierarchy. The time limitation is the connection to the GO term database.
\begin{Schunk}
\begin{Sinput}
> M <- getGo(genes,species = "human")